%!TEX TS-program = xelatex

\documentclass[a4paper,12pt, landscape]{article}

\usepackage[english,russian]{babel}   %% загружает пакет многоязыковой вёрстки
\usepackage{fontspec}      %% подготавливает загрузку шрифтов Open Type, True Type и др.
\defaultfontfeatures{Ligatures={TeX},Renderer=Basic}  %% свойства шрифтов по умолчанию
\setmainfont[Ligatures={TeX,Historic},
SmallCapsFont={Brill},
SmallCapsFeatures={Letters=SmallCaps}]{Brill} %% задаёт основной шрифт документа
\setsansfont{Brill}                    %% задаёт шрифт без засечек
\setmonofont{Noto Sans Ethiopic}
\usepackage{indentfirst}

%%% Дополнительная работа с математикой
\usepackage{amsmath,amsfonts,amssymb,amsthm,mathtools} % AMS
\usepackage{icomma} % "Умная" запятая: $0,2$ --- число, $0, 2$ --- перечисление

%%% Работа с картинками
\usepackage{wrapfig} % Обтекание рисунков текстом
\usepackage{subcaption}
\usepackage{rotating}
\usepackage{fixltx2e}
\usepackage{hhline}
\usepackage{lscape}
\usepackage[usenames,dvipsnames,svgnames,table,rgb]{xcolor}%пакет для использования цветов 

\usepackage{enumitem}
\setlist{nolistsep, leftmargin=5mm}

%%% Работа с таблицами
\usepackage{array,tabularx,tabulary,booktabs} % Дополнительная работа с таблицами
\usepackage{longtable}  % Длинные таблицы
\usepackage{multirow} % Слияние строк в таблице

\usepackage{multicol} % Несколько колонок

%%% Страница
\usepackage{extsizes} % Возможность сделать 14-й шрифт
\usepackage[twocolumn]{geometry} % Простой способ задавать поля
	\geometry{top=12mm}
	\geometry{bottom=13mm}
	\geometry{left=13mm}
	\geometry{right=9mm}

%\usepackage{fancyhdr} % Колонтитулы
% 	\pagestyle{fancy}
 	%\renewcommand{\headrulewidth}{0pt}  % Толщина линейки, отчеркивающей верхний колонтитул
% 	\lfoot{Нижний левый}
% 	\rfoot{Нижний правый}
% 	\rhead{Верхний правый}
% 	\chead{Верхний в центре}
% 	\lhead{Верхний левый}
%	\cfoot{Нижний в центре} % По умолчанию здесь номер страницы

\usepackage{setspace} % Интерлиньяж
%\onehalfspacing % Интерлиньяж 1.5
%\doublespacing % Интерлиньяж 2
\singlespacing % Интерлиньяж 1

\usepackage{lastpage} % Узнать, сколько всего страниц в документе.
\usepackage{soul} % Модификаторы начертания
\usepackage{bbding}
\usepackage{hyperref}
\usepackage[usenames,dvipsnames,svgnames,table,rgb]{xcolor}
\hypersetup{				% Гиперссылки
    colorlinks=true,       	% false: ссылки в рамках; true: цветные ссылки
    linkcolor=black,          % внутренние ссылки
    citecolor=black,        % на библиографию
    filecolor=black,      % на файлы
    urlcolor=ForestGreen          % на URL
}

\usepackage{environ}
\makeatletter
\newsavebox{\measure@tikzpicture}
\NewEnviron{scaletikzpicturetowidth}[1]{%
	\def\tikz@width{#1}%
	\def\tikzscale{1}\begin{lrbox}{\measure@tikzpicture}%
		\BODY
	\end{lrbox}%
	\pgfmathparse{#1/\wd\measure@tikzpicture}%
	\edef\tikzscale{\pgfmathresult}%
	\BODY
}
\makeatother

\usepackage{pgfplots}
\usepackage{pgfplotstable}
\usepackage{verbatim}

\usepackage{attachfile2}
 \attachfilesetup{appearance=true,
color=0 0 0
 }

%%% Лингвистические пакеты
%\usepackage{savetrees} % пакет, который экономит место
\usepackage{forest} % для рисования деревьев
\usepackage{vowel} % для рисования трапеций гласных
\usepackage{natbib}
\bibpunct[: ]{[}{]}{;}{a}{}{,}
\usepackage[nogroupskip,nopostdot, nonumberlist]{glossaries}
%\usepackage{glossary-mcols} 
%\setglossarystyle{mcolindex}
\usepackage{philex} % пакет для примеров
\addto\captionsrussian{% Replace "english" with the language you use
\renewcommand{\refname}{}
\renewcommand{\glossaryname}{Список глосс}
}
\newcommand{\mytem}{\item[$\circ$]}

\usepackage{todonotes}
\newcounter{mycomment}
\newcommand{\mycom}[1]{
\refstepcounter{mycomment}%
{%
\setstretch{0.7}% spacing
\todo[color=blue!20!white, inline]{%
\textbf{ГМ\themycomment:}~{\footnotesize #1}}%
}}
\renewcommand{\thesection}{\arabic{section}.}
\renewcommand{\thesubsection}{\arabic{section}.\arabic{subsection}}
\setlength{\columnsep}{1.6cm}

\usepackage{sectsty}
\sectionfont{\normalsize}
\subsectionfont{\normalsize}
\usepackage{titlesec}
\titlespacing*{\section}
{0pt}{2ex plus 0ex minus .2ex}{0ex plus .2ex}
\titlespacing*{\subsection}
{0pt}{2ex plus 0ex minus .2ex}{0ex plus .2ex}
\newlength{\bibitemsep}\setlength{\bibitemsep}{.2\baselineskip plus .05\baselineskip minus .05\baselineskip}
\newlength{\bibparskip}\setlength{\bibparskip}{0pt}
\let\oldthebibliography\thebibliography
\renewcommand\thebibliography[1]{%
	\oldthebibliography{#1}%
	\setlength{\parskip}{\bibitemsep}%
	\setlength{\itemsep}{\bibparskip}%
}
\begin{document}
\begin{center}{\Large Контролируемый корпус: рассказы по картинкам}
\end{center}
\begin{flushright}
	{\footnotesize Г. Мороз}
\end{flushright}
{\noindent\footnotesize последняя версия: \textbf{\href{http://1drv.ms/1NZ6zYW}{http://1drv.ms/1NZ6zYW}}}
\vspace{5mm}
\section{Введение}
\noindent Контролируемый корпус --- это корпус, собранный в результате обработки диалогов двух информантов, в ходе которого один информант рассказывает историю, основываясь на стимулах, предложенных исследователем. Традиционно в лингвистике используются визуальные стимулы (изображения или видео), но, естественно, аналогичным образом можно исследовать кодирование в языке вкусовых, тактильных или звуковых ощущений носителей или даже моделировать языковые поведение в тех или иных ситуациях. Метод контролируемого корпуса позволяет
\begin{itemize}
\mytem  избавится от основных проблем элицитации, так как влияние исследователя на порождаемое информантом значительно снижается;
\mytem значительно повысить вероятность появления исследуемых явлений, что было бы невозможно при сборе репрезентативного естественного корпуса языка.
\end{itemize}
Метод контролируемого корпуса не стоит воспринимать как аналог элицитации или корпусной лингвистике. При элицитации исследователь достаточно часто обращается скорее к знаниям носителя о языке, задавая вопросы ``Как сказать \textit{X}?'', ``Можно ли сказать \textit{X}?'', в то время как корпусные методы позволяют фиксировать случаи использования языка. Данные сущности часто достаточно сильно расходятся, что было показано в работе \citep[300]{labov64}: ``\textit{In the conscious report of their own usage, however, New York respondents are very inaccurate. <\dots> We shall see that when average New Yorkers report their own usage, they are basically giving us their norms of correctness.}''. На знания о языке достаточно часто оказывает влияние те единицы описания языка, которые используются при обучении в школе. В результате исследователь и информант оказываются в заложниках таких понятий, как \textit{предложение}, \textit{слово}, \textit{слог} и т. п. единиц письменного языка, которые редко в полной мере подходят для описания единиц звучащей речи --- language in use (см., например, \citep{miller98}). К тому же письменный и устный варианты языка могут значительно расходится друг с другом (см. \cite[40-42]{lyons68}), таким образом во время элицитации часто именно письменный язык является ``образцом'' для информанта. 
Хотя, надо полагать, в случае адыгских идиомов различие между письменным и устным вариантами не столь велико, как, скажем в случае русского или китайского.
\par Идея контролируемого корпуса не нова и восходит к работе \citep{chafe80}, однако контролируемые корпуса чаще всего используют в исследованиях дискурса. В данной работе эта методика будет использована для исследования фонологии и морфосинтаксиса.
\subsection{Сбор данных}
\noindent В каждой сессии информанты участвуют парами. Сначала каждому информанту выдается по одному набору картинок, рассказывающие две истории. Потом через какое-то время на одного из информантов надевается микрофон и он рассказывает свою историю. Второй информант его слушает, и если что-то не ясно, то спрашивает. Потом информанты меняются ролями и все повторяется. Чтобы избежать соблазна пересматривать картинки (и тем самым шуметь на аудиозаписи), картинки у рассказывающего изымаются. Как видно из описания, для данной работы необходимо подготовить минимум две истории: одну информант будет рассказывать, другую --- слушать.
\par После записи аудиозаписи аннотируются в программе ELAN, где сказанное разбирается вместе с носителями языка: происходит сегментация текста на дискурсивные единицы, создается пословный перевод, а также перевод на русский язык. Позже полученные тексты глоссируются, паузы размечаются в рамках системы нотации, предложенной в \citep{kibrik14}. Последним этапом работы с корпусом является создание необходимой аннотации для каждого исследуемого признака и извлечение целевой информации.
\subsection{Исследуемые явления}
\begin{itemize}
\mytem фонологические
\begin{itemize}
\mytem скорость речи
\mytem длительность сегментов и их составляющих
\begin{itemize}
\mytem длительность гласных
\mytem VOT стопов и аффрикат
\mytem длительность фрикативных
\end{itemize}
\mytem форманты гласных
\mytem спектральная характеристика фрикативных
\mytem чередование ɐ $\sim$ a
\item[?] исчезновение j в интервокале
\mytem \dots
\end{itemize}
\mytem грамматические
\mytem лексические
\end{itemize}
\subsection{Диалектные особенности}
\noindent При создании корпуса планируется использовать материал не только одного идиома, но хотелось бы потенциально иметь возможность собрать материал разных адыгских идиомов. В связи с этим при создании картинок, следует учесть междиалектные различия, которые позволили сделать из рассказа своего рода тест, позволяющей по набору тех или иных диалектных черт относить идиом к той или иной диалектной группе. При выборе признаков мы опирались на диалектная классификацию, представленную на рисунке \ref{dialect}.
\begin{figure}[t!]
\footnotesize
\begin{forest}
[адыгские, for tree={grow'=east, parent anchor=east, child anchor=west} 
[адыгейский [темиргоевкий [литературный (StdAd), tier = 1] [а. Псеушхо (PseuAd), tier = 1]][абадзехский (AbdzAd), tier = 1] [турецкий ад. (TurAd), tier = 1] [бжедугский (BzhAd), tier = 1][шапсугский [хакучинский (XakAd), tier = 1] [прикубанский (ShKAd), tier = 1] [причерноморский (ShBSAd), tier = 1]]]
[кабардинский [бесленеевский (BeslK), tier = 1]
[кубанский (KubK), tier = 1]
[куб.-зеленчукские г. (K-ZK), tier = 1]
[малкинский г. (MalkK), tier = 1]
[терские г. (TerK), tier = 1]
[баксанский (BaksK), tier = 1]
[турецкий кб. (TurK), tier = 1]]]
\end{forest}
\normalsize
\caption{Классификация адыгских идиомов}
\label{dialect}
\end{figure}
\par Таким образом, для составления изображений, которые потом будут использованы в исследовании, необходимо составить некоторый список фонетических, морфологических и лексических особенностей адыгских идиомов.

\section{Исследование скорость речи}
\subsection{Введение}
\noindent Судя по ссылкам, о скорости речи говорили еще в начале XX века, но первые квантитативные исследования, видимо начались с работ \citep{goldman54} и \citep{goldman56}. 
\section{Исследование лексики}

\bibliographystyle{chicago}
\bibliography{bibliography}
\end{document}